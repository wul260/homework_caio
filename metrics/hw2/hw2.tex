\documentclass{article}
\usepackage{graphicx}
\input{structure.tex} % Include the file specifying the document structure and custom commands

%----------------------------------------------------------------------------------------
%	ASSIGNMENT INFORMATION
%----------------------------------------------------------------------------------------

\title{Econometrics: Homework} % Title of the assignment

\author{Caio Figueiredo} % Author name and email address

%----------------------------------------------------------------------------------------

\begin{document}

\maketitle % Print the title

%----------------------------------------------------------------------------------------
%	PROBLEM 1
%----------------------------------------------------------------------------------------

\section{Question 1}

In order to save the bootstrap for the next question I decided to estimate
this only once, using a high $n = 10000$, the results is as follows. The data
was created using an Gumbel Distribution, for the regression estimator I used
a simple exponential function.

\begin{center}
\includegraphics[width=16cm, height=8cm]{Q1.png}
\end{center}

We can see nice that the error of the kernel estimator is very small for
low values of $h$, but as $h \rightarrow 0$ the estimator goes awry, as 
$h \rightarrow \infty$ the error increases smoothly. The same seems to
apply for the regression estimator but we can observe a second valley near $1$.

%{{{-------------------------------------------------------------------------------------
%	PROBLEM 2 
%----------------------------------------------------------------------------------------

\section{Question 2}

\subsection{a}

In the following graphs the kernel number 4 will be ignore, since it's MSE 
is much worse than the others that it undermines readability.

\begin{center}
\includegraphics[width=16cm, height=16cm]{Q2-a.png}
\end{center}

The points 1,2 and 3 are the suggested value for the bandwidth using the cross
validation method for kernel number 1, 2 and 3 respectively. The method does not
match the value of $h$ that minimizes the $MSE$ but overall behave really well and stays
in a decently close neighborhood of that point.

\subsection{b}

\begin{center}
\includegraphics[width=16cm, height=16cm]{Q2-b.png}
\end{center}

Here we have a different story, ignoring the case for $n = 10$ where the our estimator
does not behave really well the $MSE$ seems to be a increasing function of the bandwidth,
that is the lower the $MSE$ the better, which is not at all that intuitive.

Of course because of this the values suggested by the cross validation method does not
work well if the regression estimator.

This graph also have an additional line (\#5) which reference the local linear estimator,
the $MSE$ of this line does not really compare to the other lines since we are estimating
different objects but the behavior along $h$ seems to be the same.

\section{3}



\end{document}
